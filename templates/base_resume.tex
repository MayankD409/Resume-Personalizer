%-------------------------
% Resume in Latex
% Author : Jake Gutierrez
% Based off of: https://github.com/sb2nov/resume
% License : MIT
%------------------------

\documentclass[letterpaper,10pt]{article}

\usepackage{latexsym}
\usepackage[empty]{fullpage}
\usepackage{titlesec}
\usepackage{marvosym}
\usepackage[usenames,dvipsnames]{color}
\usepackage{verbatim}
\usepackage{enumitem}
\usepackage[hidelinks]{hyperref}
\usepackage{fancyhdr}
\usepackage[english]{babel}
\usepackage{tabularx}
\usepackage{fontawesome5}
\usepackage{multicol}
\usepackage{ragged2e}
\setlength{\multicolsep}{-3.0pt}
\setlength{\columnsep}{-1pt}
\input{glyphtounicode}


%----------FONT OPTIONS----------
% sans-serif
% \usepackage[sfdefault]{FiraSans}
% \usepackage[sfdefault]{roboto}
% \usepackage[sfdefault]{noto-sans}
% \usepackage[default]{sourcesanspro}

% serif
% \usepackage{CormorantGaramond}
% \usepackage{charter}


\pagestyle{fancy}
\fancyhf{} % clear all header and footer fields
\fancyfoot{}
\renewcommand{\headrulewidth}{0pt}
\renewcommand{\footrulewidth}{0pt}

% Adjust margins
\addtolength{\oddsidemargin}{-0.6in}
\addtolength{\evensidemargin}{-0.5in}
\addtolength{\textwidth}{1.19in}
\addtolength{\topmargin}{-.7in}
\addtolength{\textheight}{1.4in}

\urlstyle{same}

\raggedbottom
\raggedright
\setlength{\tabcolsep}{0in}

% Sections formatting
\titleformat{\section}{
  \vspace{-4pt}\scshape\raggedright\large\bfseries
}{}{0em}{}[\color{black}\titlerule \vspace{-5pt}]

% Ensure that generate pdf is machine readable/ATS parsable
\pdfgentounicode=1

%-------------------------
% Custom commands
\newcommand{\resumeItem}[1]{
  \item\small{
    {#1 \vspace{-2pt}}
  }
}

\newcommand{\classesList}[4]{
    \item\small{
        {#1 #2 #3 #4 \vspace{-2pt}}
  }
}

\newcommand{\resumeSubheading}[4]{
  \vspace{-2pt}\item
    \begin{tabular*}{1.0\textwidth}[t]{l@{\extracolsep{\fill}}r}
      \textbf{#1} & \textbf{\small #2} \\
      \textit{\small#3} & \textit{\small #4} \\
    \end{tabular*}\vspace{-7pt}
}

\newcommand{\resumeSubSubheading}[2]{
    \item
    \begin{tabular*}{0.97\textwidth}{l@{\extracolsep{\fill}}r}
      \textit{\small#1} & \textit{\small #2} \\
    \end{tabular*}\vspace{-7pt}
}

\newcommand{\resumeProjectHeading}[2]{
    \item
    \begin{tabular*}{1.001\textwidth}{l@{\extracolsep{\fill}}r}
      \small#1 & \textbf{\small #2}\\
    \end{tabular*}\vspace{-7pt}
}

\newcommand{\resumeSubItem}[1]{\resumeItem{#1}\vspace{-4pt}}

\renewcommand\labelitemi{$\vcenter{\hbox{\tiny$\bullet$}}$}
\renewcommand\labelitemii{$\vcenter{\hbox{\tiny$\bullet$}}$}

\newcommand{\resumeSubHeadingListStart}{\begin{itemize}[leftmargin=0.0in, label={}]}
\newcommand{\resumeSubHeadingListEnd}{\end{itemize}}
\newcommand{\resumeItemListStart}{\begin{itemize}}
\newcommand{\resumeItemListEnd}{\end{itemize}\vspace{-5pt}}

%-------------------------------------------
%%%%%%  RESUME STARTS HERE  %%%%%%%%%%%%%%%%%%%%%%%%%%%%


\begin{document}

\begin{center}
    {\Huge \scshape Mayank Deshpande} \\ \vspace{1pt}
    Santa Clara, CA, 95050 \\ \vspace{1pt}
    \small \raisebox{-0.1\height}\faPhone\ 240-481-8418 ~ \href{mailto:x@gmail.com}{\raisebox{-0.2\height}\faEnvelope\  \underline{msdeshp4@umd.edu}} ~ 
    \href{https://mayankd.me/}{\raisebox{-0.2\height}\faLink\ \underline{Website}}  ~
    \href{https://www.linkedin.com/in/msdeshpande04/}{\raisebox{-0.2\height}\faLinkedin\ \underline{LinkedIn}}  ~
    \href{https://github.com/MayankD409}{\raisebox{-0.2\height}\faGithub\ \underline{Github}} ~
    \href{https://medium.com/@deshpandemayank5}{\raisebox{-0.2\height}\faMedium\ \underline{Medium}}
    \vspace{-8pt}
\end{center}


%-----------EDUCATION-----------
\section{Education}
  \resumeSubHeadingListStart
    \resumeSubheading
      {University of Maryland, College Park}{Aug. 2023 -- Present}
      {M.Eng. Robotics, GPA: 3.97/4}{College Park, MD}
    \resumeSubheading
      {Ramdeobaba College of Engineering and Management}{Aug. 2019 -- May 2023}
      {B.E. Mechanical Engineering, GPA: 9/10}{Nagpur, IN}
  \resumeSubHeadingListEnd

%-----------PROGRAMMING SKILLS-----------
\section{Technical Skills}
 \begin{itemize}[leftmargin=0.15in, label={}]
    \small{\item{
     \textbf{Languages:}{ C++, C, Python, MATLAB} \\
     \textbf{Development Platforms:} {Linux/RTOS environments, Embedded robotics, Gazebo, AirSim, CARLA, CarSim, MoveIt, SUMO} \\
     \textbf{Libraries/Frameworks:} {ROS/ROS2, PyTorch, JAX, OpenCV, TensorFlow, Arduino, gtest, Qt, SolidWorks} \\
     \textbf{Embedded \& Architecture:} {ARM platforms, RTOS, C++\/Firmware Dev, Buses (UART, SPI, I2C, CAN), BLE, Network Protocols (MQTT?)} \\
     \textbf{Tools:} {Kubernetes, Docker, Git, Confluence, bash, nprof, Vtune, GitHub Actions, GPU Programming} \\
    }}
 \end{itemize}
 \vspace{-13pt}


%-----------EXPERIENCE-----------
\section{Experience}
  \resumeSubHeadingListStart

  \resumeSubheading
      {Intuitive Surgical Inc.}{May 2024 -- Dec 2024}
      {System Software Engineer Co-op}{Sunnyvale, CA}
      \resumeItemListStart
      \justifying
        \resumeItem{Developed an end-to-end testing pipeline in Python for the Ion Endoluminal robot, integrating real-time data logging and time-series analysis to proactively detect equipment failures. Achieved 85\% automation and reduced single-use test time by 20\%.}
        \resumeItem{Developed an optical-flow–based verification method to compare user commands with real-time video from catheter, addressing low-texture lung simulations by training a custom RAFT model and improving flow accuracy by 35\% over classical approaches.}
      \resumeItemListEnd

      \resumeSubheading
      {GAMMA AI Lab, UMD}{Jan 2024 -- May 2024}
      {Research Assistant}{College Park, MD}
      \resumeItemListStart
        \justifying
        \resumeItem{Developed a GCN based model for pedestrian trajectory prediction that inputted the robot’s planned path as future, achieving a 12\% improvement in real-time performance and maintaining sub-2ms inference time, leveling state-of-the-art ADE/FDE benchmarks.}
        \resumeItem{Validated the model on a Husky robot navigating crowded environments, confirming simulation outcomes through successful real-life experiments and demonstrating robust, low-latency trajectory forecasting.}
      \resumeItemListEnd

    \resumeSubheading
      {CodelatticeLabs Pvt. Ltd.}{May 2022 -- July 2023}
      {Robotics Software Engineer}{Bengaluru, IN}
      \resumeItemListStart
      \justifying
        \resumeItem{Developed, tested, and debugged C++ firmware for ESP32 microcontrollers, implementing communication protocols (UART, MQTT) for reliable real-time data transfer from embedded sensors.} % Added tested, debugged, reliability, protocols
        \resumeItem{Simulated and implemented multi-agent coordination algorithms and trajectory tracking methods for constrained robots, leveraging reinforcement learning for intelligent intersection management.}
      \resumeItemListEnd
    
  \resumeSubHeadingListEnd
% \vspace{-13pt}

%-----------PROJECTS-----------
\section{Projects}
    \vspace{-5pt}
    \resumeSubHeadingListStart
        \resumeProjectHeading
          {\href{https://github.com/MayankD409/RL_MPC.git}{\textbf{Adaptive RL-MPC for Autonomous Lane-Changing}} $|$ \emph{Python, SUMO, RL, MPC}}{November 2024}
          \resumeItemListStart
            \justifying
            \resumeItem{Engineered a real-time RL-MPC pipeline for autonomous lane-changing, integrating SAC/PPO/TD3 with model predictive control to dynamically adjust risk weights. Achieved 30\% higher success rates, 25\% lower collision rates, and 20\% faster lane changes in SUMO simulations, while optimizing concurrency for stable performance under varying traffic conditions.}
          \resumeItemListEnd 
          \vspace{-13pt}
          %%%%%%%%%%%%%%%%%%%%%%%%%%%%%%%%%%%%%%%%%%%%%%%%%%%%%%%%%%%%%%%%%%%%%%%%%
        \resumeProjectHeading
        {\href{https://github.com/MayankD409/HumanDetectionandTracking.git}{\textbf{Human detection and Tracking}} $|$ \emph{C++, OpenCV, MiDAS Resnet, GoogleTest, CMake}}{October 2023}
          \resumeItemListStart
            \resumeItem{Developed an C++ module for detecting and tracking human obstacles using Monocular Camera with ResNet \& OpenCV integration within the robot's reference frame.}
          \resumeItemListEnd 
          \vspace{-13pt}
          %%%%%%%%%%%%%%%%%%%%%%%%%%%%%%%%%%%%%%%%%%%%%%%%%%%%
        \resumeProjectHeading
          {\href{https://github.com/MayankD409/VisualOdom-Particle-Filter.git}{\textbf{Visual-Encoding-Particle-Filter}} $|$ \emph{C++, Python, ROS2, DL}}{May 2024}
          \resumeItemListStart
            \justifying
            \resumeItem{Developed a vision-based localization and visual odometry method for drones using a particle filter with CNN, VecKM, and Histogram of Features encoders, achieving fast convergence and real-time localization in ROS, validated in a Gazebo PX4 SITL environment.}
          \resumeItemListEnd
          \vspace{-13pt}
          %%%%%%%%%%%%%%%%%%%%%%%%%%%%%%%%%%%%%%%%%%%%%%%%%%%%%%%%%%%%%%%%%
        \resumeProjectHeading
          {\href{https://github.com/MayankD409/Structure-From-Motion.git}{\textbf{3D Reconstruction using Structure from Motion}} $|$ \emph{C++, openCV, eigen}}{March 2024}
          \resumeItemListStart
            \justifying
            \resumeItem{Developed a Structure from Motion (SfM) system using SIFT, matching, and bundle adjustment for 3D reconstruction, achieving accurate results validated with synthetic and real-world datasets.}
          \resumeItemListEnd
          \vspace{-13pt}
          %%%%%%%%%%%%%%%%%%%%%%%%%%%%%%%%%%%%%%%%%%%%%%%%%%%%%%%%
        \resumeProjectHeading
          {\href{https://github.com/MayankD409/RIEKF_Python.git}{\textbf{Right Invariant Extended Kalman Filter for object based SLAM}} $|$ \emph{Python}}{September 2023}
          \resumeItemListStart
            \justifying
            \resumeItem{Translated the theoretical RIEKF algorithm for object-based SLAM into Python, showcasing in-depth knowledge of RIEKF principles and their advantages over standard EKF in a detailed report on Yang Song et al.'s 2022 paper.}
          \resumeItemListEnd
          \vspace{-13pt}
          %%%%%%%%%%%%%%%%%%%%%%%%%%%%%%%%%%%%%%%%%%%%%%%%%%%%%%%%
        \resumeProjectHeading
          {\href{https://github.com/MayankD409/Two_Pendulum_Crane.git}{\textbf{LQR and LQG controller for two pendulum crane}} $|$ \emph{MATLAB, C++}}{October 2023}
          \resumeItemListStart
          \justifying
            \resumeItem{Modeled and controlled a two-pendulum crane system using LQR and LQG techniques in MATLAB}
          \resumeItemListEnd
          \vspace{-13pt}


        %%%%%%%%%%%%%%%%%%%%%%%%%%%%%%%%%%%%%%%%%%%%%%%%
        %%%%%%%%%%%%%% Unused Projects %%%%%%%%%%%%%%%%%
        %%%%%%%%%%%%%%%%%%%%%%%%%%%%%%%%%%%%%%%%%%%%%%%%
        % \resumeProjectHeading
        %   {\href{https://github.com/MayankD409/Humanoid-Imitation-RL.git}{\textbf{Humanoid Robot Imitation Learning from Human Videos}} $|$ \emph{Python, PyBullet, PPO, GAIL, OpenPose}}{March 2025}
        %   \resumeItemListStart
        %     \justifying
        %     \resumeItem{Developed a simulated humanoid robot that learns to accurately replicate human walking motions by combining OpenPose-based pose extraction from videos with PPO and GAIL reinforcement learning techniques, achieving stable and realistic gait patterns in PyBullet simulations.}
        %   \resumeItemListEnd
        %   \vspace{-13pt}
          
        %%%%%%%%%%%%%%%%%%%%%%%%%%%%%%%%%%%%%%%%%%%%%%%%%%%
        % \resumeProjectHeading
        %   {\href{https://github.com/MayankD409/AutoPano.git}{\textbf{AutoPano}} $|$ \emph{Python, openCV, git}}{April 2024}
        %   \resumeItemListStart
        %     \resumeItem{Developed an automatic panorama stitching solution using traditional techniques and deep learning models (HomographyNet), achieving high-quality results with supervised and unsupervised learning, validated on synthetic and real-world image sets.}
        %   \resumeItemListEnd 
        %   \vspace{-13pt}

        %%%%%%%%%%%%%%%%%%%%%%%%%%%%%%%%%%%%%%%%%%%%%%%%%%
        % \resumeProjectHeading
        %  {\textbf{Python Test Automation Framework for Embedded Device Comms} $|$ \emph{Python, pytest, UART/BLE Simulation}}{May 2024} % Adjust Date & Tech
        %  \resumeItemListStart
        %   \justifying
        %   \resumeItem{Developed a modular test automation framework in Python using pytest to verify communication protocols (simulated UART, BLE) for an embedded device target; implemented mock interfaces, automated test execution via CI/CD, and generated detailed reports, significantly improving test reliability and coverage.}
        %  \resumeItemListEnd
        %  \vspace{-13pt}

        %%%%%%%%%%%%%%%%%%%%%%%%%%%%%%%%%%%%%%%%%%%%%%%%%%
        % \resumeProjectHeading
        % {\href{https://github.com/MayankD409/AutoCalib.git}{\textbf{AutoCalib}} $|$ \emph{Python, OpenCV, NumPy, Matplotlib, CMake}}{Feb 2024}
        %   \resumeItemListStart
        %     % \resumeItem{A Python implementation of Zhang's camera calibration method, automating checkerboard detection and accurately estimating camera parameters using OpenCV and NumPy.}
        %   \resumeItemListEnd 
        %   \vspace{-13pt}

        %%%%%%%%%%%%%%%%%%%%%%%%%%%%%%%%%%%%%%%%%%%%%%%%%%
        % \resumeProjectHeading
        % {\href{https://github.com/MayankD409/Stereo-Vision.git}{\textbf{Stereo-Vision}} $|$ \emph{Python, OpenCV, NumPy, Matplotlib}}{Feb 2024}
        %   \resumeItemListStart
        %     % \resumeItem{A Python implementation of Zhang's camera calibration method, automating checkerboard detection and accurately estimating camera parameters using OpenCV and NumPy.}
        %   \resumeItemListEnd 
        %   \vspace{-13pt}

        %%%%%%%%%%%%%%%%%%%%%%%%%%%%%%%%%%%%%%%%%%%%%%%%%%
        % \resumeProjectHeading
        % {\href{https://github.com/MayankD409/Optimal_Path_Planning.git}{\textbf{Path Planning Algorithms for Point Robot}} $|$ \emph{Python, OpenCV, Numpy, Pygame}}{Feb 2024}
        %   \resumeItemListStart
        %     \resumeItem{A Python implementation of Dijkstra, Bi-RRT, and Bi-RRT* path planning algorithms for a point robot, featuring real-time visualization with Pygame and video generation using OpenCV.}
        %   \resumeItemListEnd 
        %   \vspace{-13pt}
        %%%%%%%%%%%%%%%%%%%%%%%%%%%%%%%%%%%%%%%%%%%%%%%%%%%%%%%
        % \resumeSubHeadingListStart
        % \resumeProjectHeading
        %  {\textbf{LLM Fine-Tuning for Question Answering} $|$ \emph{PyTorch, LLM Fine-Tuning, PEFT, LoRA}}{Jan 2025} % Adjust date as needed
        %  \resumeItemListStart
        %    \justifying
        %    \resumeItem{Fine-tuned a Large Language Model using Supervised Fine-Tuning and Parameter-Efficient Fine-Tuning (PEFT/LoRA) in PyTorch to significantly improve performance on domain-specific Question Answering (Financial QA).}
        %  \resumeItemListEnd
        %  \vspace{-13pt}
        %%%%%%%%%%%%%%%%%%%%%%%%%%%%%%%%%%%%%%%%%%%%%%%%%%%%%%%%%%%
        % \resumeProjectHeading
        %   {\textbf{Multi-Sensor Fusion for Robust 3D Perception} $|$ \emph{Python, C++, PyTorch, Deep Learning, Sensor Fusion}}{October 2024} % Adjust date
        %   \resumeItemListStart
        %     \justifying
        %     \resumeItem{Designed and implemented a Transformer-based deep fusion framework (PyTorch/C++) for robust 3D object detection, processing LiDAR point clouds (Open3D) and camera images to improve perception accuracy and robustness, validated in simulated adverse weather (CARLA).}
        %   \resumeItemListEnd
        %   \vspace{-13pt}

        
        
        
    \resumeSubHeadingListEnd
\vspace{1pt}


%-----------INVOLVEMENT---------------
\section{Publications}
        {\href{https://ieeexplore.ieee.org/document/9989028}{\textbf{Behavioral Analysis of ROS motion planners integrated with Robotics Middleware Framework (RMF)}} $|$ Published: 2022 $|$ \href{https://ieeexplore.ieee.org/document/9989028}{IEEE}}\\
        \textup{This paper evaluates the integration of the Robotics Middleware Framework (RMF) with Free Fleet, analyzing the performance of different path planning algorithms in multi-robot scenarios to enhance autonomous mobile robot fleet management.}


\end{document}
